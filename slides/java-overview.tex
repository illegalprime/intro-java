\documentclass{beamer}

\newcommand{\course}{CS 1331 Introduction to Object Oriented Programming}
\newcommand{\lesson}{Java Overview}

\mode<presentation>
{
  \usetheme{Berlin}
  \useoutertheme{infolines}

  % or ...

 \setbeamercovered{transparent}
  % or whatever (possibly just delete it)
}

\usepackage{hyperref}
\usepackage{fancybox}
\usepackage{listings}
\usepackage[abbr]{harvard}

\usepackage[english]{babel}
% or whatever

\usepackage[latin1]{inputenc}
% or whatever

\usepackage{times}
\usepackage[T1]{fontenc}
% Or whatever. Note that the encoding and the font should match. If T1
% does not look nice, try deleting the line with the fontenc.


\usepackage{listings}
 
% "define" Scala
\lstdefinelanguage{scala}{
  morekeywords={abstract,case,catch,class,def,%
    do,else,extends,false,final,finally,%
    for,if,implicit,import,match,mixin,%
    new,null,object,override,package,%
    private,protected,requires,return,sealed,%
    super,this,throw,trait,true,try,%
    type,val,var,while,with,yield},
  otherkeywords={=>,<-,<\%,<:,>:,\#,@},
  sensitive=true,
  morecomment=[l]{//},
  morecomment=[n]{/*}{*/},
  morestring=[b]",
  morestring=[b]',
  morestring=[b]""",
}

\usepackage{color}
\definecolor{dkgreen}{rgb}{0,0.6,0}
\definecolor{gray}{rgb}{0.5,0.5,0.5}
\definecolor{mauve}{rgb}{0.58,0,0.82}
 
% Default settings for code listings
\lstset{frame=tb,
  language=scala,
  aboveskip=3mm,
  belowskip=3mm,
  showstringspaces=false,
  columns=flexible,
  basicstyle={\scriptsize\ttfamily},
  numbers=none,
  numberstyle=\tiny\color{gray},
  keywordstyle=\color{blue},
  commentstyle=\color{dkgreen},
  stringstyle=\color{mauve},
  frame=single,
  breaklines=true,
  breakatwhitespace=true,
  keepspaces=true
  %tabsize=3
}


\title[\course] % (optional, use only with long
                                      % paper titles)
{\lesson}

\subtitle{}
%% {Include Only If Paper Has a Subtitle}

\author[Chris Simpkins] % (optional, use only with lots of authors)
{Christopher Simpkins \\\texttt{chris.simpkins@gatech.edu}}
% - Give the names in the same order as the appear in the paper.
% - Use the \inst{?} command only if the authors have different
%   affiliation.

\institute[Georgia Tech] % (optional, but mostly needed)

\date[] % (optional, should be abbreviation of
                           % conference name)
{}
% - Either use conference name or its abbreviation.
% - Not really informative to the audience, more for people (including
%   yourself) who are reading the slides online

\subject{\lesson}
% This is only inserted into the PDF information catalog. Can be left
% out. 

% If you have a file called "university-logo-filename.xxx", where xxx
% is a graphic format that can be processed by latex or pdflatex,
% resp., then you can add a logo as follows:

%% \pgfdeclareimage[width=0.6in]{gtri_logo}{gtri_logo}
%% \logo{\pgfuseimage{gtri_logo}}

% Delete this, if you do not want the table of contents to pop up at
% the beginning of each subsection:
%% \AtBeginSection[]
%% {
%%   \begin{frame}<beamer>{Outline}

%%  \tableofcontents[currentsection]
%%   \end{frame}
%% }

% If you wish to uncover everything in a step-wise fashion, uncomment
% the following command: 

% \beamerdefaultoverlayspecification{<+->}


\begin{document}

\begin{frame}
  \titlepage
\end{frame}

%------------------------------------------------------------------------
\begin{frame}[fragile]{Java}

\begin{itemize}
\item Developed for home appliances - cross-platform VM a key feature
\item Originally called Oak
\item Gained notariety with HotJava web browser that could run ``programs over the internet'' called applets
\item Gained popularity when Netscape included Java VM in Navigator web browser
\item JavaScript is purely a marketing label meant to capitalize on Java hype - there is no relationship between Java and JavaScript
\item  Java is a general-purpose application programming language.  
\item Java applets are now very rare.  The bulk of Java code runs on (web) servers.
\end{itemize}

\end{frame}
%------------------------------------------------------------------------

%------------------------------------------------------------------------
\begin{frame}[fragile]{The Java Programming Language}


\begin{itemize}
\item Java is part of the C family.  Same syntax for variable
  declarations, control structures
\item Java came at a time when C++ was king.  C++ was a notoriously complex
  object-oriented extension to C.
\item Java improved on several key aspects of C++, greatly simplifying
  software development
\item Two most compelling features of Java were cross-platform
  deployablility (``write once, run anwhere'') and autoatic garbage
  collection
\item These two advantages, especially garbage collection\footnote{In
    C and C++ the largest class of program errors were memory
    management errors.  This entire class of errors mostly disappears
    with automatic garbage collection}, drove Java adoption
\end{itemize}

\end{frame}
%------------------------------------------------------------------------

%------------------------------------------------------------------------
\begin{frame}[fragile]{The Java Platform}


Three components of the Java platform:
\begin{itemize}
\item The Java programming language
\item The Java Virtual Machine (JVM)
\item The Java standard library
\end{itemize}
Java is both compiled and interpreted:
\begin{itemize}
\item Java source files (ending in {\tt .java} are compiled to java
  bytecode files (ending in {\tt .class}
\item Java bytecode is then interpreted (run) by the JVM
\item Compiling and running can be done on different machines -
  bytecode is portable (more precisely, the JVM on each platform
  accepts the same bytecode).
\end{itemize}
The enourmous Java standard library (containing many Classes notably
missing from C++) greatly reduces software development effort.
\end{frame}
%------------------------------------------------------------------------


%------------------------------------------------------------------------
\begin{frame}[fragile]{Object-Oriented Programming in Java}


Java is an imperative programming language (we'll learn what that means next class) with support for object-oriented programming.

\begin{itemize}
\item All Java code resides within {\tt class}es
\item Classes define state (member variables) and behavior (methods)
\item Objects are instantiated from {\tt class}es (we'll see examples soon)
\item An object-oriented program models some system as a collection of objects
\item Each object communitcates with other objects by sending them messages, or {\it invoking} methods on those objects
\end{itemize}

\end{frame}
%------------------------------------------------------------------------

%------------------------------------------------------------------------
\begin{frame}[fragile]{The Anatomy of a Java Program}


Recall our {\tt HelloWorld} program:
\begin{lstlisting}[language=Java]
public class HelloWorld {
    public static void main(String[] args) {
        System.out.println("Hello, world!");
    }
}
\end{lstlisting}
\vspace{-.1in}
\begin{itemize}
\item The first line declares our {\tt HelloWorld} class.  {\tt class} is the syntax for declaring a class, and prepending with the {\tt public} modifer means the class will be visible outside {\tt HelloWorld}`s package.  For now just think of them as boilerplate.
\item Because we didn't declare a package explicitly, {\tt HelloWorld} is in the {\it default} package.  More on that in a few lectrues.
\item The code between the curly braces, {\tt \{ ... \}} define the contents of the {\tt HelloWorld} class, in this case a single method, {\tt main}
\end{itemize}

\end{frame}
%------------------------------------------------------------------------

%------------------------------------------------------------------------
\begin{frame}[fragile]{{\tt public static void main(String[] args)}}


In order to make a class executable with the {\tt java} command, it must have a main method:
\begin{lstlisting}[language=Java]
public static void main(String[] args) { ... }
\end{lstlisting}
\vspace{-.1in}
\begin{itemize}
\item The {\tt public} modifier means we can call this method from outside the class.
\item The {\tt static} modifer means the method can be called without instantiating an object of the class.  Static methods (and variables) are sometimes called {\it class} methods.
\item {\tt void} is the return type.  In particular, main returns nothing.  Sometimes such subprograms are called {\it procedures} and distinguished from {\it functions}, which return values.
\item After the method name, {\tt main}, comes the parameter list.  {\tt main} takes a single parameter of type {\tt String[]} - an array of {\tt String}s.  {\tt args} is the name of the parameter, which we can refer to within the body of {\tt main}
\end{itemize}

\end{frame}
%------------------------------------------------------------------------


%------------------------------------------------------------------------
\begin{frame}[fragile]{Compiling Java Programs}


Compile Java programs with {\tt javac}, which stands for ``Java compiler''
\begin{lstlisting}[language=Java]
$ javac HelloWorld.java
$
\end{lstlisting}
With no command line options, {\tt javac} will look in the present working directory ({\tt pwd}) for any {\tt .java} files you pass to {\tt javac} and produce corresponding {\tt .class} files.  After compiling HelloWorld.java you should have a HelloWorld.class in the same directory.

\begin{lstlisting}[language=Java]
$ ls
HelloWorld.class HelloWorld.java
$
\end{lstlisting}


\end{frame}
%------------------------------------------------------------------------

%------------------------------------------------------------------------
\begin{frame}[fragile]{Running Java Programs}


Run Java programs with {\tt java}
\begin{lstlisting}[language=Scala]
$ java HelloWorld
Hello, world!
$
\end{lstlisting}
\begin{itemize}
\item The {\tt HelloWorld} argument tells the {\tt java} command to find the {\tt .class} file named HelloWorld (which could be a file or in a JAR archive) and execute its {\tt main} method.
\end{itemize}
This is all you need to know for now.  The remainder of these slides are for the curious and motivated.
\end{frame}
%------------------------------------------------------------------------

%------------------------------------------------------------------------
\begin{frame}[fragile]{The Classpath}


To reduce clutter, you can compile classes to another directory with {\tt -d} option to
{\tt javac} 
\begin{lstlisting}[language=Java]
$ mkdir classes
$ javac -d classes HelloWorld.java
$ ls classes/
HelloWorld.class
\end{lstlisting}
Specify classpath for an application with the {\tt -cp} option to {\tt
  java}.
\begin{lstlisting}[language=Java]
$ java -cp ./classes HelloWorld
Hello, world!
\end{lstlisting}

There's a global classpath specified by the {\tt \$CLASSPATH} environment variable and a local
classpath specified by the {\tt -cp} argument to {\tt java}.  The total classpath of a Java application is the global + local classpaths.

\end{frame}
%------------------------------------------------------------------------

%------------------------------------------------------------------------
\begin{frame}[fragile]{Java Project Directory Organization}


In typical Java projects, you'll keep soruce directories and compiler
output (.class files) separate.  One simple approach is to:
\begin{itemize}
\item put source code in a {\tt src} directory, and
\item put compiler output in a {\tt classes} directory.
\end{itemize}

Let's reorganize our HelloWorld application and compile and run it.  We already have a ./classes directory.  Notice that we can compile multiple Java source files with *.java.
\begin{lstlisting}[language=Java]
$ mkdir src
$ mv HelloWorld.java src/
$ javac src/*.java -d classes/
$ ls classes
HelloWorld.class
$ java -cp ./classes HelloWorld
Hello, world!
\end{lstlisting}
\vspace{-.1in}
And now there's no source or compiler output in our working directory.  We can add classes/ to our .gitignore and forget  about them.

\end{frame}
%------------------------------------------------------------------------


% %------------------------------------------------------------------------
% \begin{frame}[fragile]{}


% \begin{lstlisting}[language=Scala]

% \end{lstlisting}


% \end{frame}
% %------------------------------------------------------------------------



% %------------------------------------------------------------------------
% \begin{frame}[fragile]{}


% \begin{lstlisting}[language=Java]

% \end{lstlisting}


% \end{frame}
% %------------------------------------------------------------------------


\end{document}
