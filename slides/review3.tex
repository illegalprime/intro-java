\documentclass{beamer}

\newcommand{\course}{CS 1331 Introduction to Object Oriented Programming}
\newcommand{\lesson}{ArrayList and Swing}
\newcommand{\code}{http://www.cc.gatech.edu/~simpkins/teaching/gatech/cs1331/code}

\author[Chris Simpkins] 
{Christopher Simpkins \\\texttt{chris.simpkins@gatech.edu}}
\institute[Georgia Tech] % (optional, but mostly needed)

\date[CS 1331]{}

\include{beamer-common}

% If you wish to uncover everything in a step-wise fashion, uncomment
% the following command: 

% \beamerdefaultoverlayspecification{<+->}


\begin{document}

\begin{frame}
  \titlepage
\end{frame}


%------------------------------------------------------------------------
\begin{frame}[fragile]{Java Collections}

\begin{center}
\includegraphics[width=4in]{colls-coreInterfaces.png}
\end{center}

{\tt ArralyList} and {\tt LinkedList} are the two basic {\tt List} implementations provided in the Java standard library.  The concepts we'll learn for {\tt ArrayList} apply to all of Java's collection classes.

\end{frame}
%------------------------------------------------------------------------

%------------------------------------------------------------------------
\begin{frame}[fragile]{{\tt ArrayList} Basics}


Create an {\tt ArrayList} with operator {\tt new}:
\begin{lstlisting}[language=Java]
  ArrayList tasks = new ArrayList();
\end{lstlisting}
Add items with {\tt add()}:
\begin{lstlisting}[language=Java]
  tasks.add("Eat");
  tasks.add("Sleep");
  tasks.add("Code");
\end{lstlisting}
Traverse with for-each loop:
\begin{lstlisting}[language=Java]
  for (Object task: tasks) {
      System.out.println(task);
  }
\end{lstlisting}

Note that the for-each loop implicitly uses an iterator.

\end{frame}
%------------------------------------------------------------------------

%------------------------------------------------------------------------
\begin{frame}[fragile]{Using Generics}


Supply a type argument in the angle brackets.  Read {\tt ArrayList<String>} as ``ArrayList of String''
\begin{lstlisting}[language=Java]
  ArrayList<String> strings = new ArrayList<String>();
  strings.add("Helluva"); strings.add("Engineer!");
\end{lstlisting}
If we try to add an object that isn't a {\tt String}, we get a compile error:
\begin{lstlisting}[language=Java]
  Integer BULL_DOG = Integer.MIN_VALUE;
  strings.add(BULL_DOG); // Won't compile
\end{lstlisting}

With a typed collection, we get autoboxing on insertion {\it and} retrieval:

\begin{lstlisting}[language=Java]
  ArrayList<Integer> ints = new ArrayList<>();
  ints.add(42);
  int num = ints.get(0);
\end{lstlisting}
Notice that we didn't need to supply the type parameter in the creation expression above.  Java inferred the type parameter from the declaration. (Note: this only works in Java 7 and above.)

See \link{\code/ArrayListGenericsDemo.java}{ArrayListGenericsDemo.java} for more.

\end{frame}
%------------------------------------------------------------------------

%------------------------------------------------------------------------
\begin{frame}[fragile]{The {\tt equals} Method and Collections}



\begin{itemize}
\item A class whose instances will be stored in a collection must have a properly implemented {\tt equals} method.
\item The {\tt contains} method in collections uses the {\tt equals} method in the stored objects.
\item The default implementation of {\tt equals} (object identity - true only for same object in memory) only rarely gives correct results.
\item Note that {\tt hashcode()} also has a defualt implementation that uses the object's memory address.  As a rule, whenever you override {\tt equals}, you should also override {\tt hashcode}
\end{itemize}


\end{frame}
%------------------------------------------------------------------------



%------------------------------------------------------------------------
\begin{frame}[fragile]{Swing}


Lots of stuff in Swing.  A few highlights:
\begin{itemize}
\item Container classes, e.g. {\tt JFrame} and {\tt JPanel}
\item Layout managers, e.g. {\tt FlowLayout}, {\tt BorderLayout}, {\tt GridLayout}
\item {\tt JLabel}s
\item {\tt Graphics} and drawing
\item Components with {\tt ActionListener}s, like {\tt JButton}
\begin{itemize}
\item Use {\tt private} inner classes
\item Avoid using a single monolithic class that extends JFrame or JPanel and implements ActionListener
\end{itemize}
\end{itemize}


\end{frame}
%------------------------------------------------------------------------

%------------------------------------------------------------------------
\begin{frame}[fragile]{The Wrong Way to Write an {\tt ActionListener}}


\begin{lstlisting}[language=Java,escapechar=`]
public class BadListener extends JFrame `\colorbox{yellow}{implements ActionListener}` {

    public BadListener() {
        super("Bad");
        setDefaultCloseOperation(EXIT_ON_CLOSE);
        JButton helloButton = new JButton("Hello");
        helloButton.addActionListener(this);
        JButton goodByeButton = new JButton("Good bye");
        goodByeButton.addActionListener(this);
        add(helloButton, BorderLayout.NORTH);
        add(goodByeButton, BorderLayout.SOUTH);
    }

    public void actionPerformed(ActionEvent e) {
        String button = `\colorbox{yellow}{e.getActionCommand()}`;
        if (button.equals("Hello")) {
            System.out.println("Hello was perssed.");
        } else if (button.equals("Good bye")) {
            System.out.println("Good bye was perssed.");
        }
    }
}
\end{lstlisting}

\end{frame}
%------------------------------------------------------------------------

%------------------------------------------------------------------------
\begin{frame}[fragile]{A Better Way to Write an {\tt ActionListener}}

\vspace{-.1in}
\begin{lstlisting}[language=Java]
public class BetterListener extends JFrame {

    private class HelloListener implements ActionListener {
        public void actionPerformed(ActionEvent e) {
            System.out.println("Hello was pressed.");
        }
    }
    private class GoodByeListener implements ActionListener {
        public void actionPerformed(ActionEvent e) {
            System.out.println("Good bye was pressed.");
        }
    }
    public BetterListener() {
        super("Better Listener");
        setDefaultCloseOperation(EXIT_ON_CLOSE);
        JButton helloButton = new JButton("Hello");
        helloButton.addActionListener(new HelloListener());
        JButton goodByeButton = new JButton("Good bye");
        goodByeButton.addActionListener(new GoodByeListener());
        add(helloButton, BorderLayout.NORTH);
        add(goodByeButton, BorderLayout.SOUTH);
    }
}
\end{lstlisting}


\end{frame}
%------------------------------------------------------------------------

% %------------------------------------------------------------------------
% \begin{frame}[fragile]{}


% \begin{lstlisting}[language=Java]

% \end{lstlisting}

% \begin{itemize}
% \item
% \end{itemize}


% \end{frame}
% %------------------------------------------------------------------------


\end{document}
