\documentclass{beamer}

\newcommand{\course}{CS 1331 Introduction to Object Oriented Programming}
\newcommand{\lesson}{Recursion}
\newcommand{\code}{http://www.cc.gatech.edu/~simpkins/teaching/gatech/cs1331/code}

\author[Chris Simpkins] 
{Christopher Simpkins \\\texttt{chris.simpkins@gatech.edu}}
\institute[Georgia Tech] % (optional, but mostly needed)

\date[CS 1331]{}

\include{beamer-common}

% If you wish to uncover everything in a step-wise fashion, uncomment
% the following command: 

% \beamerdefaultoverlayspecification{<+->}


\begin{document}

\begin{frame}
  \titlepage
\end{frame}


%------------------------------------------------------------------------
\begin{frame}[fragile]{Recursion}


\begin{itemize}
\item A recursive processes or data structure is defined in terms of itself
\item A properly written recursive function must
\begin{itemize}
\item handle the base case, and
\item convergence to the base case.
\end{itemize}
\item Failure to properly handle the base case or converge to the base case (divergence) may result in infinite recursion.
\end{itemize}


\end{frame}
%------------------------------------------------------------------------

%------------------------------------------------------------------------
\begin{frame}[fragile]{The Factorial Function}


A mathematical definition: For a non-negative integer $n$:
\begin{equation*}
    fac(n) = \begin{cases}
               1                   & \text{if } n \le 1\\
               n * fac(n-1)        & \text{otherwise}
           \end{cases}
\end{equation*}

\begin{itemize}
\item This definition tells us what a factorial is.
\item Defined in cases: a base case and a recursive case
\end{itemize}
 Factorial is defined in terms of itself

\end{frame}
%------------------------------------------------------------------------


%------------------------------------------------------------------------
\begin{frame}[fragile]{A Recursive Factorial Function}

\begin{quote}
Mathematics provides a rigorous framework for dealing with notions of {\em what is}, computation provides a rigorous framework for dealing with notions of {\em how to}. -- SICP
\end{quote}

To translate the mathematical definition of factorial (what a factorial {\it is} into a computational defiinition ({\it how to} compute a particular factorial), we need to
\begin{itemize}
\item identify the base case(s), and
\item figure out how to get our computation to converge to a base case.
\end{itemize}

For factorial, the solution is straightforward:
\begin{lstlisting}[language=Java]
public static int fac(int n) {
    if (n <= 1) {
        return 1;
    } else {
        return n * fac(n - 1);
    }
}
\end{lstlisting}

\end{frame}
%------------------------------------------------------------------------

%------------------------------------------------------------------------
\begin{frame}[fragile]{The Substitution Model of Function Evaluation}


\begin{itemize}
\item Functions are evaluated in an eval-apply cycle: function arguments are evaluated (which may in turn require function evaluation), then the function is applied to the arguments.
\item The substitution model of evaluation is a tool for understanding function evaluation in general, and recursive processes in particular.
\end{itemize}
Here's {\tt fac(5)}:
\vspace{-.05in}
\begin{lstlisting}[language=Java]
fac(5)
5 * fac(4)
5 * 4 * fac(3)
5 * 4 * 3 * fac(2)
5 * 4 * 3 * 2 * fac(1)
5 * 4 * 3 * 2 * 1
5 * 4 * 3 * 2
5 * 4 * 6
5 * 24
120
\end{lstlisting}


\end{frame}
%------------------------------------------------------------------------

%------------------------------------------------------------------------
\begin{frame}{Activation Records}

\begin{itemize}

\item Recursive subprograms cannot use statically allocated local
  variables, because each instance of the subprogram needs its own
  copies of local variables

\item Most modern languages allocate local variables for functions on
  the run-time stack.

\item The system provides a stack pointer pointing to the next
  available storage space on the stack.

\item Subprogram instances use a frame pointer that points to their
  activation record, or stack frame, which contains its copies of
  local variables

\end{itemize}

\end{frame}
%------------------------------------------------------------------------

%------------------------------------------------------------------------
\begin{frame}[fragile]{Activation Record Example}

\scriptsize
\begin{columns}
\begin{column}{3cm}
\begin{lstlisting}[language=C]
int fac(n) {
  if (n <=1) {
    return 1
  } else {
    return n * fac(n-1)
  }
}
int foo() {
  int r = 3;
  fac(r)
}
int main() {
  print(foo())
}
\end{lstlisting}
\end{column}

\begin{column}{6cm}
The stack just before fact returns with 6:
\begin{tabular}{r|c|}
 & return address to main \\ \hline
foo frame & return value (TBD) \\
 & r = 3 in foo \\
 & return address to foo \\ \hline
fac(3) frame & return value (TBD) \\
 & parameter n = 3 in fac \\
 & return address to fac \\ \hline
fac(2) frame & return value (TBD) \\
 & parameter n = 2 in fac \\
 & return address to fac \\ \hline
fac(1) frame & return value (1 by definition) \\
 & parameter n = 1 in fac \\ \hline
\end{tabular}

\end{column}
\end{columns}

\normalsize

\end{frame}
%------------------------------------------------------------------------


%------------------------------------------------------------------------
\begin{frame}[fragile]{Stack Overflow}



\begin{itemize}
\item The run-time stack is finite in size.
\item If you put too many activation records on the stack (for example by calling a recursive function with a ``large'' argument), you will overflow the stack.
\end{itemize}

\begin{lstlisting}[language=Java]
$ java Fac 10000
facLoop(10000)=0
Exception in thread "main" java.lang.StackOverflowError
	at Fac.facHelper(Fac.java:35)
	at Fac.facHelper(Fac.java:38)
	at Fac.facHelper(Fac.java:38)
...
\end{lstlisting}

Three ways to deal with this:

\begin{itemize}
\item limit input size (brittle -- how do you know limit on a particular machine?),
\item increase stack size (brittle -- how do you know how big), or 
\item replace recursion with iteration.
\end{itemize}

\end{frame}
%------------------------------------------------------------------------

%------------------------------------------------------------------------
\begin{frame}[fragile]{Looping is Imperative Recursion}


\begin{lstlisting}[language=Java]
    public static int facLoop(int n) {
        int factorial = 1;
        for (int x = n; x > 0; x--) {
            factorial *= x;
        }
        return factorial;
    }
\end{lstlisting}

\begin{itemize}
\item The base case is the termination condition for the loop.
\item The loop variable converges to the termination condition.
\end{itemize}

Recursive definitions are often more natural, but imperative/iterative definitions

\end{frame}
%------------------------------------------------------------------------


%------------------------------------------------------------------------
\begin{frame}[fragile]{Tail Recursion}

\vspace{-.05in}
\begin{lstlisting}[language=Java]
    private static int facIter(int n) {
        return facHelper(n, 1);
    }
    private static int facHelper(int n, int accum) {
        if (n <= 1) {
            return accum;
        } else {
            return facHelper(n - 1, n * accum);
        }
    }
\end{lstlisting}
\vspace{-.05in}
Tail call optimization creates an iterative, rather than a recursive process:
\vspace{-.05in}
\begin{lstlisting}[language=Java]
    facHelper(5);
    facHelper(4, 5);
    facHelper(3, 20);
    facHelper(2, 60);
    facHelper(1, 120);
    120
\end{lstlisting}
\vspace{-.05in}
Note: Java does not optimize tail calls, but many other languages (including all functional languages) do.

\end{frame}
%------------------------------------------------------------------------

%------------------------------------------------------------------------
\begin{frame}[fragile]{Closing Thoughts}


\begin{itemize}
\item Remember: A properly written recursive function must
\begin{itemize}
\item handle the base case, and
\item convergence to the base case.
\end{itemize}
\item We've seen recursive processes, next we'll learn about recursive data structures.
\end{itemize}


\end{frame}
%------------------------------------------------------------------------

% %------------------------------------------------------------------------
% \begin{frame}[fragile]{}


% \begin{lstlisting}[language=Java]

% \end{lstlisting}

% \begin{itemize}
% \item
% \end{itemize}


% \end{frame}
% %------------------------------------------------------------------------


\end{document}
