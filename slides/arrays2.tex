\documentclass{beamer}

\newcommand{\course}{CS 1331 Introduction to Object Oriented Programming}
\newcommand{\lesson}{Arrays, Part 2 of 2}
\newcommand{\code}{http://www.cc.gatech.edu/~simpkins/teaching/gatech/cs1331/code}

\author[Chris Simpkins] 
{Christopher Simpkins \\\texttt{chris.simpkins@gatech.edu}}
\institute[Georgia Tech] % (optional, but mostly needed)

\date[CS 1331]{}
\subject{\lesson}
% This is only inserted into the PDF information catalog. Can be left
% out. 

% If you have a file called "university-logo-filename.xxx", where xxx
% is a graphic format that can be processed by latex or pdflatex,
% resp., then you can add a logo as follows:

% \pgfdeclareimage[width=0.6in]{coc-logo}{cc_2012_logo}
% \logo{\pgfuseimage{coc-logo}}

\mode<presentation>
{
  \usetheme{Berlin}
  \useoutertheme{infolines}

  % or ...

 \setbeamercovered{transparent}
  % or whatever (possibly just delete it)
}

\usepackage{hyperref}
\usepackage{fancybox}
\usepackage{listings}
\usepackage[abbr]{harvard}

\usepackage[english]{babel}
% or whatever

\usepackage[latin1]{inputenc}
% or whatever

\usepackage{times}
\usepackage[T1]{fontenc}
% Or whatever. Note that the encoding and the font should match. If T1
% does not look nice, try deleting the line with the fontenc.


\usepackage{listings}
 
% "define" Scala
\lstdefinelanguage{scala}{
  morekeywords={abstract,case,catch,class,def,%
    do,else,extends,false,final,finally,%
    for,if,implicit,import,match,mixin,%
    new,null,object,override,package,%
    private,protected,requires,return,sealed,%
    super,this,throw,trait,true,try,%
    type,val,var,while,with,yield},
  otherkeywords={=>,<-,<\%,<:,>:,\#,@},
  sensitive=true,
  morecomment=[l]{//},
  morecomment=[n]{/*}{*/},
  morestring=[b]",
  morestring=[b]',
  morestring=[b]""",
}

\usepackage{color}
\definecolor{dkgreen}{rgb}{0,0.6,0}
\definecolor{gray}{rgb}{0.5,0.5,0.5}
\definecolor{mauve}{rgb}{0.58,0,0.82}
 
% Default settings for code listings
\lstset{frame=tb,
  language=scala,
  aboveskip=3mm,
  belowskip=3mm,
  showstringspaces=false,
  columns=flexible,
  basicstyle={\scriptsize\ttfamily},
  numbers=none,
  numberstyle=\tiny\color{gray},
  keywordstyle=\color{blue},
  commentstyle=\color{dkgreen},
  stringstyle=\color{mauve},
  frame=single,
  breaklines=true,
  breakatwhitespace=true,
  keepspaces=true
  %tabsize=3
}


\title[\course] % (optional, use only with long
                                      % paper titles)
{\lesson}

\subtitle{}
%% {Include Only If Paper Has a Subtitle}


% Delete this, if you do not want the table of contents to pop up at
% the beginning of each subsection:
%% \AtBeginSection[]
%% {
%%   \begin{frame}<beamer>{Outline}

%%  \tableofcontents[currentsection]
%%   \end{frame}
%% }

% If you wish to uncover everything in a step-wise fashion, uncomment
% the following command: 

% \beamerdefaultoverlayspecification{<+->}


\begin{document}

\begin{frame}
  \titlepage
\end{frame}



%------------------------------------------------------------------------
\begin{frame}[fragile]{Arrays are First-Class Objects}


We've seen that
\begin{itemize}
\item arrays are objects, so array variables are references,
\item arrays can be stored in variables, passed as arguments, and returned from methods.
\end{itemize}

This means that we must take the same precautions when dealing with array references as when dealing with references of Class types.

\end{frame}
%------------------------------------------------------------------------

%------------------------------------------------------------------------
\begin{frame}[fragile]{Array References}


Look at  \href{\code/MyYears.java}{MyYears.java}. Note that the getter method returns the private instance variable {\tt years}:
\begin{lstlisting}[language=Java]
public int[] getYears() {
    return years;
}
\end{lstlisting}

Client code cannot change the object that {\tt year} points to, but can change elements of {\tt year}.  Consider:
\begin{lstlisting}[language=Java]
private void modifyElement(int[] array, int index, int newValue) {
    array[index] = newValue;
}
\end{lstlisting}
This method will change the element value at {\tt index} in any array you pass to it, including the private {\tt years} reference returned from {\tt getYears}


\end{frame}
%------------------------------------------------------------------------


%------------------------------------------------------------------------
\begin{frame}[fragile]{Privacy Leaks with Array Instance Variables}


A getter method that returns a reference to a private array instance variable can ``leak'' private data just like an accessor that returns a reference to a private instance variable of Class type.\\
\vspace{.1in}
Given the accessor {\tt getYears} and the utility helper function {\tt modifyElement}, we can violate encapsulation in the {\tt MyYears}:
\begin{lstlisting}[language=Java]
MyYears my = new MyYears();
int[] a = my.getYears();
my.modifyElement(a, 2, 2013);
\end{lstlisting}

The code above modifies the {\tt my} object's {\tt years} instance variable!  How to fix?

\end{frame}
%------------------------------------------------------------------------

%------------------------------------------------------------------------
\begin{frame}[fragile]{Fixing Privacy Leaks}


The way to fix privacy leaks like this is to return deep copies of instance variables in getter methods.  {\tt MyYears} already has the helper method to provide this functionality.  A safer {\tt getYears} method would be:
\begin{lstlisting}[language=Java]
public int[] getYears() {
    return copyOf(years);
}
\end{lstlisting}

With this definition of {\tt getYears} client code can't modify the private data held in the {\tt years} instance variable of {\tt MyYears} objects becuase clients don't get a reference to the instance variable, they get a reference to a (snapshot) copy of it.


\end{frame}
%------------------------------------------------------------------------

%------------------------------------------------------------------------
\begin{frame}[fragile]{Partially Filled Arrays}


Say we want the speedy acces of an array, with the flixiblity of a dynamically-allocated data structure.

\begin{lstlisting}[language=Java]
public class PartialIntArray {

    private int[] elementData;
    private int size;

    public PartialIntArray(int initialCapacity) {
        if (initialCapacity < 0)
            throw new IllegalArgumentException("Illegal Capacity: "+
                                               initialCapacity);
        this.elementData = new int[initialCapacity];
    }

    public ArrayList() {
        this(10);
    }

    // ...
}
\end{lstlisting}

\end{frame}
%------------------------------------------------------------------------

%------------------------------------------------------------------------
\begin{frame}[fragile]{Adding Elements to PartialIntArray}


Assuming we don't need to automatically ``grow'' our array-backed data structure when needed (like {\tt java.util.ArrayList} does), we can add elements like this:

\begin{lstlisting}[language=Java]
public class PartialIntArray {

  // ...

    public void add(int e) {
        elementData[size++] = e;
    }

}
\end{lstlisting}

\end{frame}
%------------------------------------------------------------------------

%------------------------------------------------------------------------
\begin{frame}[fragile]{Accessing Elements of PartialIntArray}


We can provide access to elements of our PartialIntArray with:

\begin{lstlisting}[language=Java]
public class PartialIntArray {
    // ...
    
    public int get(int index) {
        return elementData(index);
    }
}
\end{lstlisting}

Note that we're providing access to individual elements, not the entire underlying array.  The underlying array is an implementation detail.
\end{frame}
%------------------------------------------------------------------------

%------------------------------------------------------------------------
\begin{frame}[fragile]{Traversing a PartialIntArray}


To allow clients of {\tt PartialIntArray} to traverse its elements, we need one more method in our API - size:
\begin{lstlisting}[language=Java]
public class PartialIntArray {
    // ...

    public int size() {
        return this.size;
    }
}
\end{lstlisting}
Now we can add elements to a {\tt PartialIntArray} and traverse it in a similar manner to regular arrays:

\begin{lstlisting}[language=Java]
PartialIntArray pia = new PartialIntArray();
pia.add(1);
// add more ...
for (int i = 0; i < pia.size(); ++i) {
    System.out.println(pia.get(i));
}
\end{lstlisting}

\end{frame}
%------------------------------------------------------------------------

%------------------------------------------------------------------------
\begin{frame}[fragile]{Encapsulation and Information Hiding}


Our {\tt PartialIntArray} class demonstrated two important concepts in software engineering: encapsulation and information hiding.

\begin{itemize}
\item The {\tt elementData} instance variable was private and never exposed in its entirety to clients.
\item All access to {\tt elementData} was provided through instance mthods.
\item We could have called our class {\tt RandomAccessIntList}, because the fact that an array was used is an implementation detail.  Client code need not be aware of implementation details (to an extent ...).
\end{itemize}

To try at home: add a {\tt remove(int index)} method that removes the element at {\tt index}.

\end{frame}
%------------------------------------------------------------------------

%------------------------------------------------------------------------
\begin{frame}[fragile]{Var-args}


\begin{lstlisting}[language=Java]

\end{lstlisting}

\begin{itemize}
\item
\end{itemize}


\end{frame}
%------------------------------------------------------------------------

%------------------------------------------------------------------------
\begin{frame}[fragile]{Enumerated Types}


\begin{lstlisting}[language=Java]

\end{lstlisting}

\begin{itemize}
\item
\end{itemize}


\end{frame}
%------------------------------------------------------------------------

%------------------------------------------------------------------------
\begin{frame}[fragile]{Multi-Dimensional Arrays}


\begin{lstlisting}[language=Java]

\end{lstlisting}

\begin{itemize}
\item
\end{itemize}


\end{frame}
%------------------------------------------------------------------------

%------------------------------------------------------------------------
\begin{frame}[fragile]{Ragged Arrays}


\begin{lstlisting}[language=Java]

\end{lstlisting}

\begin{itemize}
\item
\end{itemize}


\end{frame}
%------------------------------------------------------------------------

%------------------------------------------------------------------------
\begin{frame}[fragile]{}


\begin{lstlisting}[language=Java]

\end{lstlisting}

\begin{itemize}
\item
\end{itemize}


\end{frame}
%------------------------------------------------------------------------

%------------------------------------------------------------------------
\begin{frame}[fragile]{}


\begin{lstlisting}[language=Java]

\end{lstlisting}

\begin{itemize}
\item
\end{itemize}


\end{frame}
%------------------------------------------------------------------------

%------------------------------------------------------------------------
\begin{frame}[fragile]{}


\begin{lstlisting}[language=Java]

\end{lstlisting}

\begin{itemize}
\item
\end{itemize}


\end{frame}
%------------------------------------------------------------------------

% %------------------------------------------------------------------------
% \begin{frame}[fragile]{}


% \begin{lstlisting}[language=Java]

% \end{lstlisting}

% \begin{itemize}
% \item
% \end{itemize}


% \end{frame}
% %------------------------------------------------------------------------


\end{document}
