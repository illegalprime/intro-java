\documentclass{beamer}

\newcommand{\course}{CS 1331 Introduction to Object Oriented Programming}
\newcommand{\lesson}{Object-Oriented Programming, Part 3 of 4}
\newcommand{\code}{http://www.cc.gatech.edu/~simpkins/teaching/gatech/cs1331/code}

\author[Chris Simpkins] 
{Christopher Simpkins \\\texttt{chris.simpkins@gatech.edu}}
\institute[Georgia Tech] % (optional, but mostly needed)

\date[CS 1331]{}

\include{beamer-common}

% If you wish to uncover everything in a step-wise fashion, uncomment
% the following command: 

% \beamerdefaultoverlayspecification{<+->}


\begin{document}

\begin{frame}
  \titlepage
\end{frame}

%------------------------------------------------------------------------
\begin{frame}[fragile]{SummerIntern}


Let's add a summer intern class to our Employee hierarchy.
\begin{lstlisting}[language=Java]
public class SummerIntern extends HourlyEmployee {

    public SummerIntern(String name, Date hireDate) {
        this(name, hireDate, 20.00, 160.0);
    }
    public SummerIntern(String name, Date hireDate, 
                        double hourlyWage, double monthlyHours) {
        super(name, hireDate, hourlyWage, monthlyHours);
    }
    public double monthlyPay() {
        Calendar rightNow = Calendar.getInstance();
        return isSummer(rightNow) ? super.monthlyPay() : 0.0;
    }
    // ...
}
\end{lstlisting}

Will this compile?

\end{frame}
%------------------------------------------------------------------------


%------------------------------------------------------------------------
\begin{frame}[fragile]{The {\tt Employee} Class Hierarchy}


Now we have an expanded {\tt Employee} class hierarchy:
\vspace{-.1in}
\begin{center}
\includegraphics[height=1.5in]{expanded-employee-class-hierarchy.pdf}
\end{center}

\begin{itemize}
\item We can get the payRoll for the current month by making use of the polymorphic {\tt getMonthlyPay} method.
\item What if we wanted to get the payroll for a particular month?
\end{itemize}

Let's overload {\tt monthlyPay} so we can get the payroll for any month, not just the current month.

\end{frame}
%------------------------------------------------------------------------


%------------------------------------------------------------------------
\begin{frame}[fragile]{Overloading Methods}


An overloaded method is a set of methods with the same names but different signatures (different return types and/or parameter lists)\footnote{More precisely, two methods with the same name whose signatures are not {\it override-equivalent} are overloaded.} (\href{http://docs.oracle.com/javase/specs/jls/se7/html/jls-8.html#jls-8.4.9}{JLS \S 8.4.9}).\\
\vspace{.1in}
Here's an overloaded {\tt monthlyPay}
\begin{lstlisting}[language=Java]
public double monthlyPay() {
    Calendar rightNow = Calendar.getInstance();
    return monthlyPay(rightNow);
}

public double monthlyPay(Calendar calendar) {
    return isSummer(calendar) ? super.monthlyPay() : 0.0;
}
\end{lstlisting}

\begin{itemize}
\item In which classes should these methods be defined?
\end{itemize}


\end{frame}
%------------------------------------------------------------------------

%------------------------------------------------------------------------
\begin{frame}[fragile]{The {\tt Employee} Class Hierarchy in UML}


\vspace{-.2in}
\begin{center}
\includegraphics[height=1.5in]{employee-uml.pdf}
\end{center}
\vspace{-.25in}
\begin{itemize}
\item Italicized names are abstract (e.g., {\it Employee} is an abstract class, {\it + getMonthlyPay(month: Calendar)} is an abstract method).
\item We've only shown public methods (denoted by the '+' symbols in front of their names).
\item Each class has all the public methods in its superclasses, and possibly additional methods.
\item {\tt SummerIntern} only {\it specializes} {\tt HourlyEmployee}, that is, it modifies some behavior of its superclass but does not add any additional behavior. 
\end{itemize}


\end{frame}
%------------------------------------------------------------------------

%------------------------------------------------------------------------
\begin{frame}[fragile]{Forecasting Payroll}


Now with our overloaded  {\tt montlyPay} method we can forecast payroll:
\begin{lstlisting}[language=Java]
Company c = new Company("employees.data");
Calendar may =  Calendar.getInstance();
may.set(Calendar.MONTH, Calendar.MAY);
Calendar june = Calendar.getInstance();
june.set(Calendar.MONTH, Calendar.JUNE);
System.out.println(c.employees.get(0).monthlyPay());
System.out.printf("Monthly payroll for May: %.2f%n",
                  c.monthlyPayroll(may));
System.out.printf("Monthly payroll for June: %.2f%n",
                  c.monthlyPayroll(june));
\end{lstlisting}

Let's play with these classes.


\end{frame}
%------------------------------------------------------------------------


% %------------------------------------------------------------------------
% \begin{frame}[fragile]{}


% \begin{lstlisting}[language=Java]

% \end{lstlisting}

% \begin{itemize}
% \item
% \end{itemize}


% \end{frame}
% %------------------------------------------------------------------------


\end{document}
