\documentclass{beamer}

\newcommand{\course}{CS 1331 Introduction to Object Oriented Programming}
\newcommand{\lesson}{Binary Trees}
\newcommand{\code}{http://www.cc.gatech.edu/~simpkins/teaching/gatech/cs1331/code}

\author[Chris Simpkins] 
{Christopher Simpkins \\\texttt{chris.simpkins@gatech.edu}}
\institute[Georgia Tech] % (optional, but mostly needed)

\date[CS 1331]{}

\include{beamer-common}

% If you wish to uncover everything in a step-wise fashion, uncomment
% the following command: 

% \beamerdefaultoverlayspecification{<+->}


\begin{document}

\begin{frame}
  \titlepage
\end{frame}

%------------------------------------------------------------------------
\begin{frame}[fragile]{Binary Trees}


\begin{itemize}
\item Binary Tree Nodes
\item Binary Search Trees
\item Iterators
\end{itemize}


\end{frame}
%------------------------------------------------------------------------



%------------------------------------------------------------------------
\begin{frame}[fragile]{Binary Tree Nodes}


The nodes of a binary tree have
\begin{itemize}
\item a data item,
\item a link to a left node, and
\item a link to a right node.
\end{itemize}

\begin{lstlisting}[language=Java]
    private class Node<E> {
        E item;
        Node<E> left;
        Node<E> right;
        
        Node(E item, Node<E> left, Node<E> right) {
            this.item = item;
            this.left = left;
            this.right = right;
        }
    }
\end{lstlisting}

Just as in the other linked data structures we've studied, binary trees are recursive.


\end{frame}
%------------------------------------------------------------------------

%------------------------------------------------------------------------
\begin{frame}[fragile]{Binary Tree Structure}



\begin{itemize}
\item Every tree has a distinguished {\it root node} with no parent.
\begin{itemize}
\item All other nodes have exactly one parent.
\end{itemize}
\item Nodes which have no children are called {\it leaf nodes}.
\item Nodes which have children are called {\it interior nodes}.
\item Every node has 0, 1, or 2 children.
\item Every node can be reached by a unique path from the root node.
\end{itemize}

\begin{center}
\includegraphics[height=1.25in]{binary-tree.png}
\end{center}


\end{frame}
%------------------------------------------------------------------------

%------------------------------------------------------------------------
\begin{frame}[fragile]{Binary Search Trees}


A binary search tree (BST) is a binary tree in which for any node,  
\begin{itemize}
\item all the elements in the node's left subtree are less than the node's data item, and
\item all the elements in the node's right subtree are equal to or greater than the node's data item.
\end{itemize}

A BST is distinguished by this property, but it's ADT is simple: add elements, find element's in the tree, and iterate over the elements in a tree.


\end{frame}
%------------------------------------------------------------------------

%------------------------------------------------------------------------
\begin{frame}[fragile]{Maintaining The BST Property}


To add a new value to binary tree and maintain the BST property, we  
\begin{itemize}
\item insert new nodes for data items into the left subtree of a node if the new item is less than the node's item, or
\item the right subtree otherwise.
\end{itemize}
Every new item creates a leaf node, which can later become an interior node after additional items have been added. Here's the structure of a BST after adding the sequence of numbers 3, 4, 1, 5, 2:
\vspace{-.05in}
\begin{center}
\includegraphics[height=1.5in]{binary-tree-nums.png}
\end{center}

\end{frame}
%------------------------------------------------------------------------

%------------------------------------------------------------------------
\begin{frame}[fragile]{Traversing a Binary Tree}


There are three primary ways to traverse a binary tree:

Pre-order:
\begin{itemize}
\item Process node's item.
\item Process left subtree.
\item Process right subtree.
\end{itemize}

In-order:
\begin{itemize}
\item Process left subtree.
\item Process node's item.
\item Process right subtree.
\end{itemize}

Post-order:
\begin{itemize}
\item Process left subtree.
\item Process right subtree.
\item Process node's item.
\end{itemize}

Let's look at \link{\code/BinaryTree.java}{BinaryTree.java} for examples of these traversals.

\end{frame}
%------------------------------------------------------------------------

%------------------------------------------------------------------------
\begin{frame}[fragile]{Iterators}


Iterators can free clients from implementing such traversal algorithms.  We can even plug our data structures into Java's for-each loop by implementing {\tt java.lang.Iterable}:
\begin{lstlisting}[language=Java]
public interface Iterable<T> {
    java.util.Iterator<T> iterator();
}
\end{lstlisting}
As a reminder, {\tt java.util.Iterator}:
\begin{lstlisting}[language=Java]
public interface Iterator<E> {
    boolean hasNext();

    E next();

    void remove();
}
\end{lstlisting}

Let's implement an in-order iterator for a binary tree, a non-trivial task, to get an appreciation for the value of pre-defined iterators.

\end{frame}
%------------------------------------------------------------------------

%------------------------------------------------------------------------
\begin{frame}[fragile]{Stateful In-Order Tree Traversal}


In the traversal examples we saw earlier the traversal order was effected by the method call stack.  A stateful iterator is much more challenging because:

\begin{itemize}
\item The iterator must remember where it is in the tree
\item The iterator must be able to back-track to parent nodes after processing child branches
\end{itemize}

The essential implementation idea is to use a stack to store nodes for back-tracking.  Traditionally (at least in AI), this ``to-do list'' stack is called the {\it fringe}\footnote{This search strategy is called {\it depth-first search}.  Using a queue for the fringe instead of a stack would effect a {\it breadth-first search}}.

Let's look at \link{\code/BinaryTree.java}{BinaryTree.java} again to see how we implement a stateful in-order iterator.

\end{frame}
%------------------------------------------------------------------------

% %------------------------------------------------------------------------
% \begin{frame}[fragile]{}


% \begin{lstlisting}[language=Java]

% \end{lstlisting}

% \begin{itemize}
% \item
% \end{itemize}


% \end{frame}
% %------------------------------------------------------------------------



\end{document}
