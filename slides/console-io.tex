\documentclass{beamer}

\usepackage[english]{babel}
\usepackage[utf8]{inputenc}
\usepackage[T1]{fontenc}
\usepackage{eurosym}

\mode<presentation>
{
  \usetheme{Berlin}
  \useoutertheme{infolines}

  % or ...

 \setbeamercovered{transparent}
  % or whatever (possibly just delete it)
}
\usepackage{hyperref}
\usepackage{fancybox}
\usepackage{framed}
\usepackage{listings}
\usepackage[abbr]{harvard}

\usepackage{times}
% Or whatever. Note that the encoding and the font should match. If T1
% does not look nice, try deleting the line with the fontenc.


\usepackage{listings}
 
% "define" Scala
\lstdefinelanguage{scala}{
  morekeywords={abstract,case,catch,class,def,%
    do,else,extends,false,final,finally,%
    for,if,implicit,import,match,mixin,%
    new,null,object,override,package,%
    private,protected,requires,return,sealed,%
    super,this,throw,trait,true,try,%
    type,val,var,while,with,yield},
  otherkeywords={=>,<-,<\%,<:,>:,\#,@},
  sensitive=true,
  morecomment=[l]{//},
  morecomment=[n]{/*}{*/},
  morestring=[b]",
  morestring=[b]',
  morestring=[b]""",
}

\usepackage{color}
\definecolor{dkgreen}{rgb}{0,0.6,0}
\definecolor{gray}{rgb}{0.5,0.5,0.5}
\definecolor{mauve}{rgb}{0.58,0,0.82}
 
% Default settings for code listings
\lstset{frame=tb,
  language=scala,
  aboveskip=3mm,
  belowskip=3mm,
  showstringspaces=false,
  columns=flexible,
  basicstyle={\scriptsize\ttfamily},
  numbers=none,
  numberstyle=\tiny\color{gray},
  keywordstyle=\color{blue},
  commentstyle=\color{dkgreen},
  stringstyle=\color{mauve},
  frame=single,
  breaklines=true,
  breakatwhitespace=true,
  keepspaces=true
  %tabsize=3
}



\title[CS 1331 Introduction to Object-Oriented Programming] % (optional, use only with long
                                      % paper titles)
{Console IO}

\subtitle{}
%% {Include Only If Paper Has a Subtitle}

\author[Chris Simpkins] % (optional, use only with lots of authors)
{Christopher Simpkins \\\texttt{chris.simpkins@gatech.edu}}
% - Give the names in the same order as the appear in the paper.
% - Use the \inst{?} command only if the authors have different
%   affiliation.

\institute[Georgia Tech] % (optional, but mostly needed)

\date[] % (optional, should be abbreviation of
                           % conference name)
{}
% - Either use conference name or its abbreviation.
% - Not really informative to the audience, more for people (including
%   yourself) who are reading the slides online

\subject{Console IO}
% This is only inserted into the PDF information catalog. Can be left
% out. 

% If you have a file called "university-logo-filename.xxx", where xxx
% is a graphic format that can be processed by latex or pdflatex,
% resp., then you can add a logo as follows:

%% \pgfdeclareimage[width=0.6in]{gtri_logo}{gtri_logo}
%% \logo{\pgfuseimage{gtri_logo}}

% Delete this, if you do not want the table of contents to pop up at
% the beginning of each subsection:
%% \AtBeginSection[]
%% {
%%   \begin{frame}<beamer>{Outline}

%%  \tableofcontents[currentsection]
%%   \end{frame}
%% }

% If you wish to uncover everything in a step-wise fashion, uncomment
% the following command: 

% \beamerdefaultoverlayspecification{<+->}


\begin{document}

\begin{frame}
  \titlepage
\end{frame}


%------------------------------------------------------------------------
\begin{frame}[fragile]{Screen Output}


The Java standard library provides three primary methods in the {\tt System.out} object for sending text output to the screen.

\begin{itemize}
\item {\tt System.out.print} 
\item {\tt System.out.println} 
\item {\tt System.out.printf} (which just calls {\tt System.out.format})
\end{itemize}

\end{frame}
%------------------------------------------------------------------------

%------------------------------------------------------------------------
\begin{frame}[fragile]{{\tt System.out.print}}


{\tt System.out.print} takes a {\tt String} parameter and sends the string to the screen.  The statements
\begin{lstlisting}[language=Java]
System.out.print("Me");
System.out.print("ow!");
\end{lstlisting}
will produce the output
\begin{lstlisting}[language=Java]
Meow!
\end{lstlisting}

\end{frame}
%------------------------------------------------------------------------

%------------------------------------------------------------------------
\begin{frame}[fragile]{{\tt System.out.println}}


{\tt System.out.println} does the same as {\tt System.out.print} but adds a newline character.  The statements
\begin{lstlisting}[language=Java]
System.out.println("Johnny");
System.out.println("Chimpo");
\end{lstlisting}
will produce the output
\begin{lstlisting}[language=Java]
Johnny
Chimpo
\end{lstlisting}

\end{frame}
%------------------------------------------------------------------------

%------------------------------------------------------------------------
\begin{frame}[fragile]{{\tt System.out.printf}}


{\tt System.out.printf} takes a {\it format string} and any number of additional arguments, and prints the result of inserting the additional arguments into the format string according to the format specifiers in the format string
\begin{itemize}
\item The format string can contain other text in addition to format specifiers
\item Each format specifier begins with {\tt \%} and ends with a {\it conversion character}
\item You can think of each format specifier as defining a field into which a value is inserted
\item Like {\tt print}, {\tt printf} does not print a newline character at the end.  End your format string with {\tt \\n} if you want to end your output with a newline
\end{itemize}
 {\tt printf} is a convenience method for {\tt format}
\end{frame}
%------------------------------------------------------------------------

%------------------------------------------------------------------------
\begin{frame}[fragile]{{\tt System.out.printf} Examples}


For full details, see \url{http://docs.oracle.com/javase/7/docs/api/java/util/Formatter.html#syntax}.  Here are a few examples
\begin{itemize}

\item ``Decimals'' (integers) - {\tt d}, Strings - {\tt s}
\begin{lstlisting}[language=Java]
System.out.printf("%d %s.\n", 7, "Samurai");
\end{lstlisting}
prints
\begin{lstlisting}[language=Java]
7 Samurai
\end{lstlisting}

\item Floating point numbers - {\tt f}
\begin{lstlisting}[language=Java]
 System.out.printf("I like %3.2f.%n", Math.PI);
\end{lstlisting}
prints
\begin{lstlisting}[language=Java]
I like 3.14
\end{lstlisting}

\end{itemize}

Play around with \url{http://www.cc.gatech.edu/~simpkins/teaching/gatech/cs1331/code/ConsoleOutput.java} to get a feel for printf.

\end{frame}
%------------------------------------------------------------------------

%------------------------------------------------------------------------
\begin{frame}[fragile]{Number Formatting}


{\tt printf} is useful for general formatting, but if you need to print currency amounts and you want to ``internationalize'' your code, use a CurrencyFormatter {\tt NumberFormat}.

\begin{lstlisting}[language=Java]
NumberFormat us = NumberFormat.getCurrencyInstance();
System.out.println(us.format(3.14));

NumberFormat de = NumberFormat.getCurrencyInstance(Locale.GERMANY);
System.out.println(de.format(3.14));     
\end{lstlisting}
prints\\
\begin{framed}
\$3.14\\
3,14 \euro
\end{framed}
 


\end{frame}
%------------------------------------------------------------------------

%------------------------------------------------------------------------
\begin{frame}[fragile]{Packages and Imports}
\vspace{-.1in}
\begin{itemize}
\item All Java classes are organized in packages
\item We've been using the default package (by not specifying a package)
\item To use a class from a different package, you must either fully qualify it every time you use it, or import it
\end{itemize}
{\tt NumberFormat} is in the {\tt java.text} package.  The top of the {\tt NumberFormat} class contains the line:
\begin{lstlisting}[language=Java]
package java.text;
\end{lstlisting}
 And {\tt Locale} is in the {\tt java.util} package.  So for our example from the previous slide to work we must include the following import statements at the top of our source file:
\begin{lstlisting}[language=Java]
import java.text.NumberFormat;
import java.util.Locale;
\end{lstlisting}
\vspace{-.1in}
See \url{http://www.cc.gatech.edu/~simpkins/teaching/gatech/cs1331/code/CurrencyFormatting.java}
\end{frame}
%------------------------------------------------------------------------

%------------------------------------------------------------------------
\begin{frame}[fragile]{Console Input}


You can read input from the console using the {\tt Scanner} class
\begin{itemize}
\item First import it from the {\tt java.util} package
\begin{lstlisting}[language=Java]
import java.util.Scanner;
\end{lstlisting}
\item Then you can use a {\tt Scanner} object to read, for example, three integers like this:
\begin{lstlisting}[language=Java]
Scanner keyboard = new Scanner(System.in);
System.out.println("Enter your 3 test scores, separated by spaces.");
exam1 = keyboard.nextInt();
exam2 = keyboard.nextInt();
exam3 = keyboard.nextInt();
examAvg = (exam1 + exam2 + exam3) / 3.0; // Why 3.0 instead of 3?
System.out.printf("Your exam average is %.1f%n", examAvg);\end{lstlisting}
\end{itemize}
\end{frame}
%------------------------------------------------------------------------

%------------------------------------------------------------------------
\begin{frame}[fragile]{Basic File Input using {\tt Scanner}}


You can read form a file the same way you read form a keyboard by simply initializing with a {\tt FileInputStream} instead of {\tt System.in}
\begin{lstlisting}[language=Java]
Scanner gradeFile = new Scanner(new FileInputStream("grades.txt"));
\end{lstlisting}
 
See \url{http://www.cc.gatech.edu/~simpkins/teaching/gatech/cs1331/code/CourseAverage.java} for an example.
\end{frame}
%------------------------------------------------------------------------

% %------------------------------------------------------------------------
% \begin{frame}[fragile]{}

% \begin{itemize}

% \item 
% \begin{lstlisting}[language=Java]

% \end{lstlisting}
 
% \end{itemize}

% \end{frame}
% %------------------------------------------------------------------------

\end{document}
